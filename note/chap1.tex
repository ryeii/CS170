\section{Big-O Notation}

\begin{definition}
Let $f(n)$ and $g(n)$ be functions from positive integers to positive reals. We say $f = O(g)$ if there is a constant $c>0$ such that $f(n)\le cg(n)$
\end{definition}
Saying $f = O(g)$ is a very loose analog of “$f \le g$.” 
\begin{definition}
\[
f=\Omega (g) \Longleftrightarrow g=O(f)
\]
\[
f=\Theta (g) \Longleftrightarrow f=O(g)\;\land\;f=\Omega(g) 
\]
\end{definition}
Saying $f = \Omega(g)$ is a very loose analog of "$f \ge g$,"
and therefore $f = \Theta(g)$ means that $f$ and $g$ takes, in average, the time to run as the input size grows (g encloses f both from above and below).


\begin{example}
\huge TODO
\end{example}